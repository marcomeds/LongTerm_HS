\documentclass[aspectratio=169,numbering=none]{beamer}
\usefonttheme{default}
\setbeamerfont{frametitle}{size=\large}
\usepackage{amsmath,amssymb,mathrsfs}
\usepackage{multirow}
\usepackage{color}
\usepackage{pdfpages}
\usepackage{tabu}
\usepackage{bbm}
\usepackage{graphicx}
\usepackage{epstopdf}
\usepackage{bigstrut}
\usepackage{color,array}
\usepackage{hyperref}
\usepackage{array}
\usepackage{pdflscape}
\usepackage{booktabs}
\usepackage{rotating}
\usepackage{threeparttable}
\usepackage{booktabs}
\usepackage{xcolor,colortbl}
\usepackage{apacite}
\usepackage{caption}
\usepackage{subcaption}
\usepackage{tikz}
\usepackage{multicol}

\usepackage{tabu} %rowfont color

\usetikzlibrary{positioning}
\tikzset{mynode/.style={draw,align=center}}

\newtheorem{assumption}{Assumption}




\DeclareCaptionFont{blue}{\color{LightSteelBlue3}}
\newcommand{\foo}{\color{LightSteelBlue3}\makebox[0pt]{\textbullet}\hskip-0.5pt\vrule width 1pt\hspace{\labelsep}}

\newenvironment{wideitemize}{\itemize\addtolength{\itemsep}{10pt}}{\enditemize}

\newcolumntype{L}[1]{>{\raggedright\let\newline\\\arraybackslash\hspace{0pt}}m{#1}}
\newcolumntype{C}[1]{>{\centering\let\newline\\\arraybackslash\hspace{0pt}}m{#1}}
\newcolumntype{R}[1]{>{\raggedleft\let\newline\\\arraybackslash\hspace{0pt}}m{#1}}

\def\zapcolorreset{\let\reset@color\relax\ignorespaces}
\def\colorrows#1{\noalign{\aftergroup\zapcolorreset#1}\ignorespaces}


%center text in tables
\newcommand{\specialCellCenter}[2][c]{\begin{tabular}[#1]{@{}c@{}}#2\end{tabular}}
\newcommand{\specialcell}[2][l]{\begin{tabular}[#1]{@{}l@{}}#2\end{tabular}}

\newcolumntype{H}{>{\setbox0=\hbox\bgroup}c<{\egroup}@{}}
\newcommand\independent{\protect\mathpalette{\protect\independenT}{\perp}}
\def\independenT#1#2{\mathrel{\rlap{$#1#2$}\mkern2mu{#1#2}}}
\newenvironment{changemargin}[2]{%
  \begin{list}{}{%
    \setlength{\topsep}{0pt}%
    \setlength{\leftmargin}{#1}%
    \setlength{\rightmargin}{#2}%
    \setlength{\listparindent}{\parindent}%
    \setlength{\itemindent}{\parindent}%
    \setlength{\parsep}{\parskip}%
  }%
  \item[]}{\end{list}}


\mode<presentation>
\definecolor{important}{rgb}{0.1,0.1,0.9}

%Mauro defined
\makeatletter\let\expandableinput\@@input\makeatother
\makeatletter
%primitive input in tabular
\AddToHook{env/tabular/begin}{\let\input\@@input}
\makeatother

\DeclareMathOperator*{\argmin}{argmin}
\renewcommand{\hat}{\widehat}

\setbeamertemplate{navigation symbols}{} 

\def\sym#1{\ifmmode^{#1}\else\(^{#1}\)\fi}
\definecolor{myblue}{rgb}{.1,.4,.5}
\title{\textcolor{myblue}{COVID-19 Learning loss and recovery: \\
Panel data evidence from India}}


%\author{RISE Programme: India Research Team }
\author{\large{Abhijeet Singh}\thanks{Stockholm School of Economics; J-PAL; E-mail:  \href{mailto:abhijeet.singh@hhs.se}{abhijeet.singh@hhs.se}
}
\and
\large{Mauricio Romero}\thanks{ITAM; J-PAL; E-mail: \href{mtromero@itam.mx}{mtromero@itam.mx}}
\and
\large{Karthik Muralidharan}\thanks{UC San Diego; NBER; J-PAL; E-mail: \href{mailto:kamurali@ucsd.edu}{kamurali@ucsd.edu}}
}
%\institute{\begin{multicols}{2}
%Mauricio Romero (ITAM)\\
%Abhijeet Singh (Stockholm School of Economics)\\
%\end{multicols}}


%new fontsize
\makeatletter
\newcommand\notsotiny{\@setfontsize\notsotiny\@vipt\@viipt}
\makeatother

%TIMELINE
\usepackage{tikz}
\usepackage{environ}
\makeatletter
\newsavebox{\measure@tikzpicture}
\NewEnviron{scaletikzpicturetowidth}[1]{%
  \def\tikz@width{#1}%
  \def\tikzscale{1}\begin{lrbox}{\measure@tikzpicture}%
  \BODY
  \end{lrbox}%
  \pgfmathparse{#1/\wd\measure@tikzpicture}%
  \edef\tikzscale{\pgfmathresult}%
  \BODY
}
\makeatother


\makeatletter
\AddToHook{env/tabular/begin}{\let\input\@@input}
\AddToHook{env/threeparttable/begin}{\let\input\@@input}
\AddToHook{env/table/begin}{\let\input\@@input}
\makeatother

\setbeamertemplate{note page}[compress]
\setbeamerfont{note page}{size=\tiny}
\setbeameroption{hide notes}
%\setbeameroption{show only notes}

\setbeamersize{text margin left=0.2in,text margin right=0.2in}


\normalsize
\date{\vspace{1cm} \textcolor{gray}{NBER Education Program meeting}\\  December 9, 2022}
\begin{document}
\frame[plain]{\titlepage}

\section{Introduction}
\setcounter{tocdepth}{1}
%\mode<beamer>{  %  Options for use in slides only.
%    \AtBeginSection[] {
%    \setcounter{tocdepth}{2}
%      \begin{frame}<beamer>{The incidence and effects of affirmative action}
%        \tableofcontents
%      \end{frame}
%    }
%}
\mode<beamer>{  %  Options for use in slides only.
    \AtBeginSection[] {
    \setcounter{tocdepth}{1}
      \begin{frame}<beamer>{Learning loss (and recovery)}
        \tableofcontents[currentsection]
      \end{frame}
    }
    \AtBeginSubsection[] {
    \setcounter{tocdepth}{2}
      \begin{frame}<beamer>{Learning loss (and recovery)}
        \tableofcontents[currentsection,currentsubsection]
      \end{frame}
    }
}

\begin{frame}[plain]{Motivation }
\begin{itemize}
\vfill\item COVID-19 disrupted education systems worldwide, \textbf{but shock more severe in LMICs}
\begin{itemize}
\vfill\item LMICs had longer school closures  \cite{imf_school_closures, unesco_school_closures}
\vfill\item Schools/parents in LMICs less equipped to pivot to remote instruction
\vfill\item LMICs more vulnerable to economic and health shocks \cite{patrinos2022analysis}
\vfill\item May exacerbate `learning crisis' and increase educational inequality \cite{world2020covid}
\vfill\item Lost lifetime earnings due to reduced human capital accumulation estimated at up to $\sim$\$17 trillion  \cite{world2021covid}
\end{itemize}
\pause
\vfill\item India illustrates the severity of the COVID-19 shock in LMICs starkly
\begin{itemize}
\vfill\item Most of the economy reopened much before schools did (schools were closed $\sim$ 18 months)
\vfill\item An estimated 3.2 million people died by Sept 2021 \cite{jha2022covid}
\vfill\item Severe household economic shocks, esp during lockdowns \cite{kesar2021pandemic}
\end{itemize}
\pause
\vfill\item Substantial (justified) effort in understanding ``learning loss''
\begin{itemize}
\vfill\item Relatively little evidence (yet) on magnitudes \cite{Pratham2021CG}
\vfill\item It is very unclear how persistent these losses are (or how quickly children converge)
\vfill \item Very little evidence on how to address system-wide losses \emph{after schools have reopened}

\end{itemize}
\end{itemize}
\end{frame}

\begin{frame}[plain]{This paper}
\begin{itemize}
\vfill \item We provide new evidence for Tamil Nadu (pop. $\sim$77 million)
\begin{itemize}
\vfill\item Large panel data set: household-based census across 220 villages, $\sim$19k children
\vfill\item Multiple rounds of in-person data collection (2019, Dec 2021, Feb 2022 and May 2022)
\vfill\item Near-representative of rural communities in the state
\end{itemize}
\pause
\vfill \item Focus on students of preschool and primary schooling age
\begin{itemize}
\vfill\item The group most relevant for Foundational Literacy and Numeracy goals
\end{itemize}
\pause
\vfill \item \textbf{Three main research questions}:
\begin{itemize}
\vfill\item How large were ``learning losses’’ at the end of school closures?
\vfill\item How much did children recover in the first six months of school re-opening?
\vfill\item How much of the recovery comes from a large-scale government remediation program?

\end{itemize}
\vfill \item For each of these questions, we'll also look at inequality
\end{itemize}
\end{frame}

\begin{frame}[plain]{Contributions to the literature}
\begin{itemize}
\vfill \item COVID-19 learning losses in LMICs \cite{patrinos2022analysis,moscoviz2022learning}
\begin{itemize}
\vfill \item Most estimates rely on simulations or phone-based testing in non-representative samples (except \citeA{hevia2022estimation} in Mexico)
\vfill \item[$\Rightarrow$] \textbf{\color{blue}{In-person testing in a near-representative sample, IRT-linked measurement}}
% \vfill \item Only one of 36 studies reviewed by \citeA{patrinos2022analysis} features representative samples with in-person testing (\citeA{hevia2022estimation} in Mexico)
\end{itemize}
\vfill \item Evidence on recovery
\begin{itemize}
\vfill \item[$\Rightarrow$] \textbf{\color{blue}{Measure system-wide catch-up in LMICs}} 
\vfill \item[$\Rightarrow$] Other studies? \cite{lichand2022lasting}
\end{itemize}
\vfill \item Evidence on mitigation/remediation
\begin{itemize}
\vfill \item Several studies on remote tutoring and technology interventions \textit{during} school closures (e.g., \citeA{angrist2022experimental, carlana2021apart,hassan2021telementoring})
\vfill \item[$\Rightarrow$] \textbf{\color{blue}{Evidence on at-scale program to remediate learning losses upon school opening}}
\end{itemize}
\end{itemize}
\end{frame}


\begin{frame}<beamer>{Learning loss (and recovery)}
        \tableofcontents
 \end{frame}
      
\section{Setting}
\begin{frame}[plain,label=setting]{Setting}
\begin{itemize}
\vfill\item In Jun-Aug of 2019: census of households with children $<8$ yrs as the baseline for an RCT
\begin{itemize}
\vfill\item 220 villages are spread across 4 districts in Tamil Nadu
\vfill\item 21,046 households
\vfill\item 25,126 children 2--7 years old tested in math and Tamil
\vfill\item Households resemble rural TN aggregates in state-wide representative data \hyperlink{representative}{\beamergotobutton{Comparison}}
\end{itemize}
\vfill\item The experimental treatment was abandoned due to the COVID-19 pandemic
\vfill\item Between Dec 2021 to May 2022, we went back to resurvey all the households
\begin{itemize}
\vfill\item Tested all children 3--10 years old
\vfill\item 19,289 children aged 4--10 years for whom we have test scores in 2019 and 2021/22
\vfill\item  Attrition does not vary by gender, SES, or baseline test scores \hyperlink{attrition}{\beamergotobutton{Attrition}}
\end{itemize}
% \vfill\item Follow-up data collection was done in 3 stages:
% \begin{itemize}
% \vfill\item Wave 1 (after schools reopen): Dec 20, 2021-Jan 7, 2022 (N=7,683)
% \vfill\item Wave 2 (after Omicron): Feb 25 –March 23, 2022 (N=5616)
% \vfill\item Wave 3 (5 months after reopening): March 11 – May 7 (N=13,798)
% \end{itemize}
\end{itemize}
\end{frame}

\begin{frame}[plain]{Map of sample districts in Tamil Nadu}
\centering
\includegraphics[width=0.5\textwidth]{Figures/Districts.pdf}
\end{frame}






\begin{frame}[plain,label=timing]{Measurement Waves}
\begin{itemize}
\vfill\item We randomized households into two groups within each village \hyperlink{balance_app}{\beamergotobutton{Balance}}
\begin{itemize}
   \vfill \item These were designated the ``early" and ``late'' follow-up groups
   \vfill \item The ``early" group was (arbitrarily) split into two sub-groups due to the Omicron wave 
   \vfill \item Generates 2 rounds of student-level measurement, and 4 rounds of cohort-level mean scores  
\end{itemize}
\vfill\item Due to Omicron wave, follow-up data collection was done in 3 stages:
\begin{enumerate}
\vfill\item \textbf{Early follow-up I}: Dec 20th, 2021 -- Jan 7th,2022 (N=7,683)
\vfill\item \textbf{Early follow-up II}: Feb 25 -- March 23,2022 (N=5,616)
\vfill\item \textbf{Late follow-up}: March 11 -- May 7, 2022 (N=13,798)
\end{enumerate}
\vfill\item Variation between early and late follow-up is random (stratified within village)
\vfill \item Variation within early group not randomized, but balanced on observables
\end{itemize}
\end{frame}

\begin{frame}[plain]{Timeline}
\begin{figure}[H]

 \begin{center}
    \begin{scaletikzpicturetowidth}{\textwidth}
\begin{tikzpicture}[scale=\tikzscale]
        % draw horizontal line   
        \draw (-12,0) -- (25,0);
    
    % draw vertical lines
    \foreach \x in {-11,-10,-9,-8,-7,-6,-5,-4,-3,-2,-1,0,1,2,3,4,5,6,7,8,9,10,11,12,13,14,15,16,17,18,19,20,21,22,23,24}
    \draw (\x cm,3pt) -- (\x cm,-3pt);
    
    % draw nodes
    \draw (-11,0) node[below=3pt] {\begin{turn}{90} Jun/19 \end{turn}} node[above=3pt] {$\begin{turn}{90}
            Baseline begins
        \end{turn}$};
    \draw (-10,0) node[below=3pt] {\begin{turn}{90} Jul/19 \end{turn}} node[above=3pt] {$\begin{turn}{90}
        \end{turn}$};
    \draw (-9,0) node[below=3pt] {\begin{turn}{90} Aug/19 \end{turn}} node[above=3pt]  {$\begin{turn}{90}
            Baseline ends
        \end{turn}$};
    \draw (-8,0) node[below=3pt] {\begin{turn}{90}  Sep/19 \end{turn}} node[above=3pt] {$  $};
    \draw (-7,0) node[below=3pt] {\begin{turn}{90} Oct/19 \end{turn}} node[above=3pt] {$  $};
    \draw (-6,0) node[below=3pt] {\begin{turn}{90} Nov/19 \end{turn}} node[above=3pt] {$  $};
    \draw (-5,0) node[below=3pt] {\begin{turn}{90} Dec/19 \end{turn}} node[above=3pt] {$\begin{turn}{90}
            First COVID-19 case in Wuhan
        \end{turn}$};
    \draw (-4,0) node[below=3pt] {\begin{turn}{90} Jan/20 \end{turn}} node[above=3pt] {$  $};
    \draw (-3,0) node[below=3pt] {\begin{turn}{90} Feb/20 \end{turn}} node[above=3pt] {$  $};
    \draw (-2,0) node[below=3pt] {\begin{turn}{90} Mar/20 \end{turn}} node[above=3pt]  {$\begin{turn}{90}
            Schools close in Tamil Nadu
        \end{turn}$};
    \draw (-1,0) node[below=3pt] {\begin{turn}{90} Apr/20 \end{turn}} node[above=3pt] {$  $};
    \draw (0,0) node[below=3pt] {\begin{turn}{90} May/20 \end{turn}} node[above=3pt] {$  $};
    \draw (1,0) node[below=3pt] {\begin{turn}{90} Jun/20 \end{turn}} node[above=3pt] {$  $};
    \draw (2,0) node[below=3pt] {\begin{turn}{90} Jul/20 \end{turn}} node[above=3pt] {$  $};
    \draw (3,0) node[below=3pt] {\begin{turn}{90} Aug/20 \end{turn}} node[above=3pt] {$\begin{turn}{90}
        \end{turn}$};
    \draw (4,0) node[below=3pt] {\begin{turn}{90} Sep/20 \end{turn}} node[above=3pt] {$\begin{turn}{90}
        \end{turn}$};
    \draw (5,0) node[below=3pt] {\begin{turn}{90} Oct/20 \end{turn}} node[above=3pt] {$  $};
    \draw (6,0) node[below=3pt] {\begin{turn}{90} Nov/20 \end{turn}} node[above=3pt] {$\begin{turn}{90}
        \end{turn}$};
    \draw (7,0) node[below=3pt] {\begin{turn}{90} Dec/20 \end{turn}} node[above=3pt] {$  $};
    \draw (8,0) node[below=3pt] {\begin{turn}{90} Jan/21 \end{turn}} node[above=3pt] {$\begin{turn}{90}
        \end{turn}$};
    \draw (9,0) node[below=3pt] {\begin{turn}{90} Feb/21 \end{turn}} node[above=3pt] {$  $};
    \draw (10,0) node[below=3pt] {\begin{turn}{90} Mar/21 \end{turn}} node[above=3pt] {$  $};
    \draw (11,0) node[below=3pt] {\begin{turn}{90} Apr/21 \end{turn}} node[above=3pt] {$  $};
    \draw (12,0) node[below=3pt] {\begin{turn}{90}
    May/21 \end{turn}} node[above=3pt] {$  $};
    \draw (13,0) node[below=3pt] {\begin{turn}{90}
    Jun/21 \end{turn}} node[above=3pt] {$  $};
    \draw (14,0) node[below=3pt] {\begin{turn}{90} Jul/21 \end{turn}} node[above=3pt] {$  $};
    \draw (15,0) node[below=3pt] {\begin{turn}{90} Aug/21 \end{turn}} node[above=3pt] {$  $};
    \draw (16,0) node[below=3pt] {\begin{turn}{90} Sep/21 \end{turn}} node[above=3pt] {$  $};
    \draw (17,0) node[below=3pt] {\begin{turn}{90} Oct/21 \end{turn}} node[above=3pt] {$  $};
    \draw (18,0) node[below=3pt] {\begin{turn}{90} Nov/21 \end{turn}} node[above=3pt] {$ \begin{turn}{90} Schools re-open in Tamil Nadu
    \end{turn} $};
    \draw (19,0) node[below=3pt] {\begin{turn}{90} Dec/21 \end{turn}} node[above=3pt] {$\begin{turn}{90}
    1st round of follow-up begins
    \end{turn}$};
    \draw (20,0) node[below=3pt] {\begin{turn}{90}
    Jan/22 \end{turn}} node[above=3pt] {$\begin{turn}{90} 
    Schools shut (Omicron)/R1 end
    \end{turn}$};
    \draw (21,0) node[below=3pt] {\begin{turn}{90} Feb/22 \end{turn}} node[above=3pt] {$\begin{turn}{90} 
    2nd round of follow-up begins
    \end{turn}$};
    \draw (22,0) node[below=3pt] {\begin{turn}{90} Mar/22 \end{turn}} node[above=3pt] {$\begin{turn}{90} 
    2nd round ends/3rd begins
    \end{turn}$};
    \draw (23,0) node[below=3pt] {\begin{turn}{90} Apr/22 \end{turn}} node[above=3pt] {$  $};
    \draw (24,0) node[below=3pt] {\begin{turn}{90}
    May/22 \end{turn}} node[above=3pt] {$\begin{turn}{90}
    3rd round ends
    \end{turn}$};
\end{tikzpicture}
\end{scaletikzpicturetowidth}
\end{center} 
\end{figure}
\end{frame}

\begin{frame}[plain]{Difference in observed characteristics across rounds}
\centering
\resizebox{0.8\textheight}{!}{
\begin{tabular}{lccccc}
\toprule
&   (1) 	&  (2)	&  (3) 	& (4)	& (5)	\\
& Dec/21- & Feb/22- & Mar/22-  & $p$-value & $p$-value  \\
& Jan/22 &  Mar/22& May/22 &  ($H_o$: Equality) & ($H_o$: Equality   \\
& & & & & within village) \\
\midrule
\input{Tables/Table_a2_villageFE}
\bottomrule
\end{tabular}
}
\end{frame}





\begin{frame}[plain]{Learning Assessments}
\begin{itemize}
\vfill \item Our focus was on measuring foundational literacy and numeracy
\begin{itemize}
   \vfill \item In 2019, these covered simple competences (in keeping with a focus on kids aged 3-5)
   \vfill \item In 2021, we broadened the range of skills to be able to measure ages 3-10 
\end{itemize}
\pause
\vfill \item Tests are scored using Item Response Theory
\begin{itemize}
   \vfill \item Large overlap of items between rounds and ages
   \vfill \item Tests are vertically linked across rounds and linked across ages
   \vfill \item Test scores standardized with reference to 5-year-olds in 2019
\end{itemize}
\pause
\vfill \item Tests are typically well-distributed (esp in 2021/22)
\begin{itemize}
   \vfill \item In 2019, problems of floor/ceiling effects at ends of age distribution
   \vfill \item Fixed in 2021, with a broader range of items
   \vfill \item No results are sensitive to this
\end{itemize}
\end{itemize}
\end{frame}




\section{Quantifying Learning Loss (and recovery)}

\begin{frame}[plain]{Learning profile in math in 2019}
\begin{figure}[H]
\begin{center}
\includegraphics[width=0.75\textwidth]{Figures/LL_Math_1_2.pdf} 
\end{center}
\end{figure}
\end{frame}

\begin{frame}[plain]{Large drop in age-adjusted achievement in Dec 2021}
\framesubtitle{...which is a change in \textit{gradient}}
\begin{figure}[H]
\begin{center}
\includegraphics[width=0.75\textwidth]{Figures/LL_Math_2_2.pdf} 
\end{center}
\end{figure}
\end{frame}

\begin{frame}[plain]{Large drop in age-adjusted achievement in Dec 2021}
\framesubtitle{...which is a change in \textit{gradient}}
\begin{figure}[H]
\begin{center}
\includegraphics[width=0.75\textwidth]{Figures/LL_Math_2_vertical_2.pdf} 
\end{center}
\end{figure}
\end{frame}

\begin{frame}[plain]{Large drop in age-adjusted achievement in Dec 2021}
\framesubtitle{...which is a change in \textit{gradient}}
\begin{figure}[H]
\begin{center}
\includegraphics[width=0.75\textwidth]{Figures/LL_Math_2_horizontal_2.pdf} 
\end{center}
\end{figure}
\end{frame}

\begin{frame}[plain]{But this gap is smaller by Feb 2022}
\begin{figure}[H]
\begin{center}
\includegraphics[width=0.75\textwidth]{Figures/LL_Math_3_2.pdf} 
\end{center}
\end{figure}
\end{frame}

\begin{frame}[plain]{Rapid recovery: 2/3 of the initial gap gone by April '22}
\begin{figure}[H]
\begin{center}
\includegraphics[width=0.75\textwidth]{Figures/LL_Math_4_2.pdf} 
\end{center}
\end{figure}
\end{frame}

\begin{frame}[plain]{Catch-up seems as rapid in later primary school}
\framesubtitle{At ages 8-10 (where we don't have a baseline)}
\begin{figure}[H]
\begin{center}
\includegraphics[width=0.75\textwidth]{Figures/LL_Math_all_ages_2.pdf} 
\end{center}
\end{figure}
\end{frame}


\begin{frame}[plain]{Rapid recovery also in language (with smaller initial losses)}
\begin{figure}[H]
\begin{center}
\includegraphics[width=0.75\textwidth]{Figures/LL_Tamil_2.pdf} 
\end{center}
\end{figure}
\end{frame}

\begin{frame}{Learning loss at different ages}
\begin{table}
\centering
    \caption{Learning loss between August 2019 and December 2021}
\resizebox{\textwidth}{!}{
    \begin{tabular}{lcccccccc}
    \toprule
  & (1) & (2) & (3) & (4) & (5) & (6) & (7) & (8) \\
\midrule
\multicolumn{4}{l}{\textbf{Panel A: Learning loss at different ages}}\\
&	\multicolumn{4}{c}{Math}&	\multicolumn{4}{c}{Tamil} \\
 \cmidrule(lr){2-5} \cmidrule(lr){6-9}
Age (in months) &  60  & 72  & 84  & 96  &  60  & 72  & 84  & 96  \\
\\
IRT score (Aug 2019) & \input{Tables/W0_60_Math} & \input{Tables/W0_72_Math} & \input{Tables/W0_84_Math} & \input{Tables/W0_96_Math} & \input{Tables/W0_60_Tamil} & \input{Tables/W0_72_Tamil} & \input{Tables/W0_84_Tamil} & \input{Tables/W0_96_Tamil}   \\
IRT score (Dec 2021) & \input{Tables/W1_60_Math} & \input{Tables/W1_72_Math} & \input{Tables/W1_84_Math} & \input{Tables/W1_96_Math} & \input{Tables/W1_60_Tamil} & \input{Tables/W1_72_Tamil} & \input{Tables/W1_84_Tamil} & \input{Tables/W1_96_Tamil}   \\
Absolute loss (in SD) &   \input{Tables/Loss60_Math} &  \input{Tables/Loss72_Math}  &  \input{Tables/Loss84_Math} & \input{Tables/Loss96_Math} &  \input{Tables/Loss60_Tamil} &  \input{Tables/Loss72_Tamil}  &  \input{Tables/Loss84_Tamil} & \input{Tables/Loss96_Tamil} \\
Developmental lag (in months) &   \input{Tables/DevLag60_Math} &  \input{Tables/DevLag72_Math} &
\input{Tables/DevLag84_Math} &
\input{Tables/DevLag96_Math}&  \input{Tables/DevLag60_Tamil} &  \input{Tables/DevLag72_Tamil}  &  \input{Tables/DevLag84_Tamil} & \input{Tables/DevLag96_Tamil}\\
\bottomrule
\end{tabular}
}
\footnotesize
\raggedright \textit{Notes}: Panel A presents, for children of different ages, the raw IRT score in wave 0 (Aug 2019) and wave 1 (Dec 2021), as well as the difference between the two (the absolute learning loss in standard deviations), and the developmental lag (i.e., how much longer, in months, it took a student in 2021 to achieve the same score as a student in 2019). 

\end{table}
\end{frame}


\begin{frame}[plain]{Heterogeneity in learning loss}
\centering
\resizebox{\textwidth}{!}{
    \begin{tabular}{lcccccccc}
    \toprule
  & (1) & (2) & (3) & (4) & (5) & (6) & (7) & (8) \\
\midrule
&	\multicolumn{4}{c}{Math score (in SD)}&	\multicolumn{4}{c}{Tamil score (in SD)} \\
 \cmidrule(lr){2-5} \cmidrule(lr){6-9}
\input{Tables/Table_LearningLoss_Nocov}
\bottomrule
\end{tabular}
}
\raggedright \footnotesize{Note: This table compares Wave 0 (Aug 2019) to Wave 1 (Dec 2021). Village FE included in all regressions. Standard errors clustered at village level.}
\end{frame}

\begin{frame}{Recovery from learning loss}
\begin{table}[H]
\centering
\footnotesize
\centering
    \begin{tabular}{lcccccccc}
    \toprule
  & (1) & (2) & (3) & (4) & (5) & (6) & (7) & (8) \\
\midrule
\multicolumn{4}{l}{\textbf{Panel A: Recovery at different ages}}\\
&	\multicolumn{4}{c}{Math}&	\multicolumn{4}{c}{Tamil} \\
 \cmidrule(lr){2-5} \cmidrule(lr){6-9}
Age (in months) &  60  & 72  & 84  & 96  &  60  & 72  & 84  & 96  \\
\\
IRT score (Aug 2019) & \input{Tables/W0_60_Math} & \input{Tables/W0_72_Math} & \input{Tables/W0_84_Math} & \input{Tables/W0_96_Math} & \input{Tables/W0_60_Tamil} & \input{Tables/W0_72_Tamil} & \input{Tables/W0_84_Tamil} & \input{Tables/W0_96_Tamil}   \\
IRT score (Dec 2021) & \input{Tables/W1_60_Math} & \input{Tables/W1_72_Math} & \input{Tables/W1_84_Math} & \input{Tables/W1_96_Math} & \input{Tables/W1_60_Tamil} & \input{Tables/W1_72_Tamil} & \input{Tables/W1_84_Tamil} & \input{Tables/W1_96_Tamil}   \\
IRT score (Feb 2022) & \input{Tables/W2_60_Math} & \input{Tables/W2_72_Math} & \input{Tables/W2_84_Math} & \input{Tables/W2_96_Math} & \input{Tables/W2_60_Tamil} & \input{Tables/W2_72_Tamil} & \input{Tables/W2_84_Tamil} & \input{Tables/W2_96_Tamil}   \\
IRT score (Apr 2022) & \input{Tables/W3_60_Math} & \input{Tables/W3_72_Math} & \input{Tables/W3_84_Math} & \input{Tables/W3_96_Math} & \input{Tables/W3_60_Tamil} & \input{Tables/W3_72_Tamil} & \input{Tables/W3_84_Tamil} & \input{Tables/W3_96_Tamil}   \\
& & & & & & & & & \\
Absolute loss (in SD) &   \input{Tables/Loss60_Math} &  \input{Tables/Loss72_Math}  &  \input{Tables/Loss84_Math} & \input{Tables/Loss96_Math} &  \input{Tables/Loss60_Tamil} &  \input{Tables/Loss72_Tamil}  &  \input{Tables/Loss84_Tamil} & \input{Tables/Loss96_Tamil} \\
Absolute recovery (in SD) by Feb 22 &   \input{Tables/RecW2_60_Math} &  \input{Tables/RecW2_72_Math}  &  \input{Tables/RecW2_84_Math} & \input{Tables/RecW2_96_Math} &  \input{Tables/RecW2_60_Tamil} &  \input{Tables/RecW2_72_Tamil}  &  \input{Tables/RecW2_84_Tamil} & \input{Tables/RecW2_96_Tamil} \\
Absolute recovery (in SD) by Apr 22 &   \input{Tables/RecW3_60_Math} &  \input{Tables/RecW3_72_Math}  &  \input{Tables/RecW3_84_Math} & \input{Tables/RecW3_96_Math} &  \input{Tables/RecW3_60_Tamil} &  \input{Tables/RecW3_72_Tamil}  &  \input{Tables/RecW3_84_Tamil} & \input{Tables/RecW3_96_Tamil} \\
\bottomrule
\end{tabular} 
\end{table}
\footnotesize
\raggedright \textit{Notes}: Panel A presents, for children of different ages, the raw IRT score in wave 0 (Aug 2019) and three survey waves in 2021-22. The difference between the Aug '19 and Dec '21 waves measures absolute learning loss. The difference between Dec '21 and the subsequent rounds measures recovery. 
\end{frame}

\begin{frame}[plain]{Heterogeneity in recovery from learning loss}
\centering
\resizebox{0.7\textwidth}{!}{
    \begin{tabular}{lcccccccc}
    \toprule
  & (1) & (2) & (3) & (4) & (5) & (6) & (7) & (8) \\
\midrule
&	\multicolumn{4}{c}{Math score (in SD)}&	\multicolumn{4}{c}{Tamil score (in SD)} \\
 \cmidrule(lr){2-5} \cmidrule(lr){6-9}
\input{Tables/Table_LearningRecovery_NoCov_slides}
\bottomrule
\end{tabular}
}

\raggedright \footnotesize{Note: This table compares Wave 1 (Dec 2021) to Waves 2 (Feb 2022) and 3 (April 2022). Village FE included in all regressions. Standard errors clustered at village level.}
\end{frame}




\begin{frame}[plain]{Is recovery an artifact of what/how we test?}
\begin{itemize}
   \vfill \item \textbf{Does recovery reflect the material being tested?}
    \begin{itemize}
        \item Tests are not linked to syllabus
        \item Content largely about foundational literacy and numeracy
        \item Esp. in 2021-22, these tests are designed with broad coverage
    \end{itemize}
    \vfill \item \textbf{Does recovery reflect mode of administration?}
    \begin{itemize}
        \item Tests are administered at home in a 1:1 setting
        \item Not just mechanically caused by, e.g., greater familiarity with writing 
    \end{itemize}
    \vfill \item \textbf{Does recovery reflect other failures in testing or linking?}
    \begin{itemize}
        \item Very similar patterns when looking at individual items and percent correct
        \item Ceiling/floor effects not an issue in the 2021/22 rounds
    \end{itemize}
\end{itemize}
\end{frame}

\begin{frame}[plain,label=catchup]{How should we think about catch-up?}
\begin{itemize}
\vfill \item The recovery likely a result of many factors
\begin{itemize}
\vfill\item ``Natural’’ recovery as students return to school
\vfill\item  Compensatory actions taken at school or by households
\vfill\item \textbf{State Government started a large supplementary instruction program for recovery}
\end{itemize}
\end{itemize}
\end{frame}




\section{Ilam Thedi Kalvi (``Education at doorstep'')}




\begin{frame}[plain]{Massive state-wide after-school remedial campaign}
\framesubtitle{Ilam Thedi Kalvi (``Education at doorstep'')}
\begin{itemize}
\vfill \item The state government's main initiative is called Ilam Thedi Kalvi (ITK)
\begin{itemize}
   \vfill  \item Started in Nov 2021, extended broadly in January 2022
    \vfill \item 60-90 minutes of supplementary lessons 
   \vfill  \item Delivered by cadre of volunteers (more than 200,000 in the state)
    \vfill  \item Stipend of INR 1,000/month ($\sim$ 12 USD) ---  vs. INR 28,660/month for primary teachers
    \vfill \item Held in various spaces (schools, AWCs, volunteer house)
    \vfill \item Special curriculum developed to remedy learning loss
   \vfill  \item Very salient in the state
\end{itemize}
\pause
\vfill \item Similar to \citeA{banerjee2017proof} and \citeA{duflo2020experimental}
\vfill\item In the last survey wave (April-May wave), we elicit extensive questions about ITK
\end{itemize}
\end{frame}


\begin{frame}[plain]{Households know about ITK}
\framesubtitle{And report their children attend}
\begin{itemize}
    \vfill \item 91\% of households have heard about ITK
    \vfill \item 57\% of HH report children attend ITK
    \vfill \item Of those attending, 92\% HH report	$\geq$ 4 days per week attendance
    \vfill \item Attendance rates are higher among poorer households
\begin{itemize}
    \vfill \item Most likely reflects implementation led by government schools
    \vfill \item Could also reflect lack of alternative means to support recovery
\end{itemize}
\end{itemize}
\end{frame}


\begin{frame}[plain,label=itk_progressive]{ITK participation was progressive}
\centering
\resizebox{0.8\textheight}{!}{
\begin{tabular}{lcccc}
\toprule
&   (1) 	&  (2)	&  (3) 	&  (4)	\\
& Does not & Attend  & Difference  & Difference \\
& attend ITK & ITK & (overall)  & (village FE) \\
\midrule
\input{Tables/SelectionITK_1}
\bottomrule
\end{tabular}
}
\end{frame}

\begin{frame}[plain,label=itk_progressive]{ITK participation was concentrated among government school students}
\centering
\resizebox{\textheight}{!}{
\begin{tabular}{lcccc}
\toprule
&   (1) 	&  (2)	&  (3) 	&  (4)	\\
& Does not & Attend  & Difference  & Difference \\
& attend ITK & ITK & (overall)  & (village FE) \\
\midrule
\input{Tables/SelectionITK_2}
\bottomrule
\end{tabular}
}
\begin{itemize}
    \item Program messaging and recruitment were all through the government machinery
\end{itemize}
\end{frame}

\begin{frame}[plain,label=itk_progressive]{ITK participants received fewer inputs to cope during COVID-19 school closures}
\centering
\resizebox{\textheight}{!}{
\begin{tabular}{lcccc}
\toprule
&   (1) 	&  (2)	&  (3) 	&  (4)	\\
& Does not & Attend  & Difference  & Difference \\
& attend ITK & ITK & (overall)  & (village FE) \\
\midrule
\input{Tables/InputsITK_1}
\bottomrule
\end{tabular}
}
\begin{itemize}
    \item Also lower use of internet, smartphones, books at home etc.
\end{itemize}
\end{frame}


\begin{frame}[plain]{Evaluating the causal effect of ITK}
\begin{itemize}
\vfill \item ITK participants seem negatively selected in the population
    \vfill \item We use ``value-added’’ models (VAM) which control for baseline scores, demographics, enrollment type (determined before program rollout) and village FE
\begin{itemize}
    \item  These rely on conditional ignorability for identification
\end{itemize}

\pause

\vfill \item Even without exogenous variation, estimates likely to approximate causal effect
\begin{itemize}

\vfill \item VAM typically yield similar estimates as RCTs, RD, DiD, substantial predictive validity
\begin{itemize}
    \vfill\item School effects \cite{andrabi2011value,deming2014using,singh2015private,angrist2017leveraging,singh2020learning,angrist2021credible}
    \vfill\item Teacher effects \cite{chetty2014measuring, bacher2014validating, bau2020teacher}
\end{itemize}
\end{itemize}
\pause
\vfill \item We will further control for extensive direct inputs as well, a la Chetty et al (2014)
\vfill \item Estimate Oster (2019) bounds




\end{itemize}
\end{frame}

\begin{frame}[plain]{Evaluating the causal effect of ITK}

\begin{equation}
  Y_{it} = \alpha_{v} + \beta.AttendITK_{it} + \gamma.\mathbf{X_i} + \phi.\mathbf{Y_{i,t-1}} +\epsilon_{it} 
\end{equation}
\begin{itemize}
\vfill \item  $Y_{it}$: achievement in 2022
\vfill \item  $AttendITK_{it}$: indicator for whether child $i$ attends an ITK center
\vfill \item  $\alpha_v$: vector of village-level dummy variables
\vfill \item  $\mathbf{X_i}$: background characteristics (age, gender, SES, maternal education, and enrollment)
\vfill \item  $\mathbf{Y_{i,t-1}}$: lagged achievement measures in math and Tamil in 2019
\vfill \item   $\epsilon_{it}$: error term
\end{itemize}
\end{frame}


\begin{frame}[plain]{Assessing effect of \textit{Illam Thedi Kalvi (ITK)} on math test scores}
\small
    \begin{tabular}{>{\onslide<1->}l>{\onslide<2->}c>{\onslide<3->}c>{\onslide<4->}c>{\onslide<5->}c>{\onslide<6->}c>{\onslide<7->}c}
\toprule  
  & (1) & (2) & (3) & (4) & (5)   \\
\midrule
&	\multicolumn{1}{c}{Naive}&	\multicolumn{1}{c}{VAM} &	\multicolumn{3}{c}{Augmented} \\
 \cmidrule(lr){2-2} \cmidrule(lr){3-3}  \cmidrule(lr){4-6}
\input{Tables/Table_ITKEffect_Math}
\bottomrule 
\end{tabular}

\hyperlink{itk_sensitivity_full}{\beamergotobutton{Full table}}
\end{frame}

\begin{frame}[plain,label=itk_effect]{Assessing effect of \textit{Illam Thedi Kalvi (ITK)} on Tamil test scores}
\small
    \begin{tabular}{lcccccc}
    \toprule
  & (1) & (2) & (3) & (4) & (5)   \\
\midrule
&	\multicolumn{1}{c}{Naive}&	\multicolumn{1}{c}{VAM} &	\multicolumn{3}{c}{Augmented} \\
 \cmidrule(lr){2-2} \cmidrule(lr){3-3}  \cmidrule(lr){4-6}
\input{Tables/Table_ITKEffect_Tamil}
\bottomrule
\end{tabular}

\hyperlink{itk_sensitivity_full}{\beamergotobutton{Full table}}
\end{frame}

\subsubsection{Sensitivity to further omitted variables bias}\label{sec:robust2}
\begin{frame}[plain]{Oster bounds}
\begin{itemize}
    \vfill \item Sensitivity of our results to further omitted variables bias \cite{oster2019}
    \begin{itemize}
\vfill \item Assume that selection-on-unobservables equals selection on observed variables

       \vfill \item Note; given negative selection into ITK, this will raise effect sizes
       \vfill \item We'll treat age and village FE as orthogonal (base specifications) 

    \vfill \item Keep in mind: even the rich vector of inputs raises $R^2$ by 0.01
        \end{itemize}
\end{itemize}
\end{frame}


\begin{frame}[plain]{Sensitivity of Math \textit{Illam Thedi Kalvi (ITK)} estimates to omitted variables bias}
    \begin{tabular}{lccccHHHH}
    \toprule
$R^2_{max}=$  & $\tilde R^2 +0.1(\tilde R^2-\mathring R^2)$   & $\tilde R^2 +0.3(\tilde R^2-\mathring R^2)$  & $\tilde R^2 +0.5(\tilde R^2-\mathring R^2)$   & $\tilde R^2 +0.7(\tilde R^2-\mathring R^2)$   & $\tilde R^2 +0.9(\tilde R^2-\mathring R^2)$   & $\tilde R^2 +1(\tilde R^2-\mathring R^2)$  & $\tilde R^2 +1.1(\tilde R^2-\mathring R^2)$  & $\tilde R^2 +1.3(\tilde R^2-\mathring R^2)$ \\ 
  & (1) & (2) & (3) & (4) & (5) & (6) & (7) & (8) \\
\midrule
\multicolumn{3}{l}{\textbf{Panel A: Math}}\\
$\beta^*$ & \input{Tables/MathOster_01R_Age} & \input{Tables/MathOster_03R_Age} & \input{Tables/MathOster_05R_Age} & \input{Tables/MathOster_07R_Age} & \input{Tables/MathOster_09R_Age} & \input{Tables/MathOster_1R_Age} & \input{Tables/MathOster_11R_Age} & \input{Tables/MathOster_13R_Age} \\
\multicolumn{3}{l}{\textbf{}}\\
$\mathring\beta$ & \input{Tables/MathOster_Beta0_Age} & \input{Tables/MathOster_Beta0_Age} & \input{Tables/MathOster_Beta0_Age} & \input{Tables/MathOster_Beta0_Age} & \input{Tables/MathOster_Beta0_Age} & \input{Tables/MathOster_Beta0_Age} & \input{Tables/MathOster_Beta0_Age} & \input{Tables/MathOster_Beta0_Age}  \\
$\tilde{\beta}$ & \input{Tables/MathOster_BetaTidle_Age} & \input{Tables/MathOster_BetaTidle_Age} & \input{Tables/MathOster_BetaTidle_Age} & \input{Tables/MathOster_BetaTidle_Age} & \input{Tables/MathOster_BetaTidle_Age} & \input{Tables/MathOster_BetaTidle_Age} & \input{Tables/MathOster_BetaTidle_Age} & \input{Tables/MathOster_BetaTidle_Age} \\
$\mathring R^2$ & \input{Tables/MathOster_R0_Age} & \input{Tables/MathOster_R0_Age} & \input{Tables/MathOster_R0_Age} & \input{Tables/MathOster_R0_Age} & \input{Tables/MathOster_R0_Age} & \input{Tables/MathOster_R0_Age} & \input{Tables/MathOster_R0_Age} & \input{Tables/MathOster_R0_Age} \\
$\tilde  R^2$ & \input{Tables/MathOster_RTidle_Age} & \input{Tables/MathOster_RTidle_Age} & \input{Tables/MathOster_RTidle_Age} & \input{Tables/MathOster_RTidle_Age} & \input{Tables/MathOster_RTidle_Age} & \input{Tables/MathOster_RTidle_Age} & \input{Tables/MathOster_RTidle_Age} & \input{Tables/MathOster_RTidle_Age} \\
\bottomrule
\end{tabular}
\end{frame}

\begin{frame}[plain]{Sensitivity of Tamil \textit{Illam Thedi Kalvi (ITK)} estimates to omitted variables bias}
    \begin{tabular}{lccccHHHH}
    \toprule
$R^2_{max}=$  & $\tilde R^2 +0.1(\tilde R^2-\mathring R^2)$   & $\tilde R^2 +0.3(\tilde R^2-\mathring R^2)$  & $\tilde R^2 +0.5(\tilde R^2-\mathring R^2)$   & $\tilde R^2 +0.7(\tilde R^2-\mathring R^2)$   & $\tilde R^2 +0.9(\tilde R^2-\mathring R^2)$   & $\tilde R^2 +1(\tilde R^2-\mathring R^2)$  & $\tilde R^2 +1.1(\tilde R^2-\mathring R^2)$  & $\tilde R^2 +1.3(\tilde R^2-\mathring R^2)$ \\ 
  & (1) & (2) & (3) & (4) & (5) & (6) & (7) & (8) \\
\midrule
\multicolumn{3}{l}{\textbf{Panel B: Tamil}}\\
$\beta^*$ & \input{Tables/TamilOster_01R_Age} & \input{Tables/TamilOster_03R_Age} & \input{Tables/TamilOster_05R_Age} & \input{Tables/TamilOster_07R_Age} & \input{Tables/TamilOster_09R_Age} & \input{Tables/TamilOster_1R_Age} & \input{Tables/TamilOster_11R_Age} & \input{Tables/TamilOster_13R_Age} \\
\multicolumn{3}{l}{\textbf{}}\\
$\mathring\beta$ & \input{Tables/TamilOster_Beta0_Age} & \input{Tables/TamilOster_Beta0_Age} & \input{Tables/TamilOster_Beta0_Age} & \input{Tables/TamilOster_Beta0_Age} & \input{Tables/TamilOster_Beta0_Age} & \input{Tables/TamilOster_Beta0_Age} & \input{Tables/TamilOster_Beta0_Age} & \input{Tables/TamilOster_Beta0_Age}  \\
$\tilde{\beta}$ & \input{Tables/TamilOster_BetaTidle_Age} & \input{Tables/TamilOster_BetaTidle_Age} & \input{Tables/TamilOster_BetaTidle_Age} & \input{Tables/TamilOster_BetaTidle_Age} & \input{Tables/TamilOster_BetaTidle_Age} & \input{Tables/TamilOster_BetaTidle_Age} & \input{Tables/TamilOster_BetaTidle_Age} & \input{Tables/TamilOster_BetaTidle_Age} \\
$\mathring R^2$ & \input{Tables/TamilOster_R0_Age} & \input{Tables/TamilOster_R0_Age} & \input{Tables/TamilOster_R0_Age} & \input{Tables/TamilOster_R0_Age} & \input{Tables/TamilOster_R0_Age} & \input{Tables/TamilOster_R0_Age} & \input{Tables/TamilOster_R0_Age} & \input{Tables/TamilOster_R0_Age} \\
$\tilde  R^2$ & \input{Tables/TamilOster_RTidle_Age} & \input{Tables/TamilOster_RTidle_Age} & \input{Tables/TamilOster_RTidle_Age} & \input{Tables/TamilOster_RTidle_Age} & \input{Tables/TamilOster_RTidle_Age} & \input{Tables/TamilOster_RTidle_Age} & \input{Tables/TamilOster_RTidle_Age} & \input{Tables/TamilOster_RTidle_Age} \\
\bottomrule
\end{tabular}
\end{frame}

\subsubsection{Heterogeneity in ITK program effects}

\begin{frame}[plain]{ITK appears to contribute to the progressivity of cohort-level learning recovery}
\centering
\resizebox{1.3\textheight}{!}{
    \begin{tabular}{lcccccccc}
    \toprule
&	\multicolumn{4}{c}{Math}&	\multicolumn{4}{c}{Tamil} \\
 \cmidrule(lr){2-5} \cmidrule(lr){6-9}
  & (1) & (2) & (3) & (4) & (5) & (6) & (7) & (8) \\
\midrule
\input{Tables/Table_ITKEffect_Hetero_slides}
\bottomrule
\end{tabular}
}
\begin{itemize}
    \item Most of the contribution is from progressive participation, not differential effectiveness
\end{itemize}
\end{frame}


\begin{frame}[plain]{Estimating the contribution of ITK to recovery from learning losses}
\begin{itemize}
\vfill \item ITK effect of 0.17 standard deviations in Math (for participants)
\begin{itemize}
\vfill \item Compared to a learning loss of 0.67 SD in December 2021
\vfill \item Compared to a recovery of 0.45 SD between Dec 2021 - May 2022
\vfill \item Contribution to cohort-level catch up is 0.097 SD (0.17 $\times$ 57\% take up)
\vfill \item $\sim$ 20\% of the \textit{population-level} catch-up 
\end{itemize}
\vfill \item ITK effect of 0.093  standard deviations in Tamil (for participants)
\begin{itemize}
    \vfill \item Compared to a learning loss of 0.33 SD in December 2021
    \vfill \item Compared to a recovery of  0.19 SD between Dec 2021 and May 2022
    \vfill \item Contribution to cohort-level catch up is 0.053 SD (0.093 $\times$ 57\% take up)
    \vfill \item $\sim$ 28\% of the \textit{population-level} catch-up 
\end{itemize}
\vfill \item $\sim$ 50\% of the \textit{population-level} learning loss would have been made up even without ITK
\end{itemize}
\end{frame}


\section{Conclusion}

\begin{frame}[plain]{What can we take away from this?}
\begin{itemize}
\vfill \item The worry about substantial learning losses is not misplaced
\begin{itemize}
\vfill\item \textbf{Huge learning losses} at the point of school re-openings (Dec '2021)
\end{itemize}
\vfill \item However, these losses do not have to be permanent
\begin{itemize}
\vfill\item Fast recovery ---  2/3 of gap closed in 4 months!
\vfill\item Reopening schools accounted for $\sim$ 50\% recovery
\vfill \item Supplemental remedial instruction can accelerate recovery and compensate regressive losses
\end{itemize}
\vfill \item \textbf{Understanding effects of COVID-19 on education will need long-term follow-ups}
\begin{itemize}
\vfill\item Current estimates of learning loss only partially informative
\end{itemize}
\vfill \item \underline{\textbf{Caution:}} Fast recovery might not be a ``structural'' feature
\begin{itemize}
\vfill\item See e.g. \citeA{andrabi2021human} on the Pakistan 2005 earthquake
\vfill\item Recovery likely reflects pandemic response policies and behavior 
\vfill\item Recovery may well be slower elsewhere (or absent!) 
\end{itemize}
\end{itemize}
\end{frame}

\begin{frame}[plain]{Intended follow-ups}
\begin{itemize}
 \vfill \item \textbf{Does recovery persist?}
 \begin{itemize}
     \item We will resurvey these children in Feb-March 2023
     \item Was there further catch-up?
     \item Eventually, how far behind are students compared to pre-pandemic trajectories?
 \end{itemize}
 \vfill  \item \textbf{Does ITK continue to look effective?}
\begin{itemize}
   \item Do we see any additional gains from ITK?
    \item Did student and volunteer engagement continue after the initial pandemic emergency?
    \item Could this be a complement to the universal FLN campaigns of the government?
\end{itemize}
 \vfill \item  \textbf{What are the effects of becoming an ITK volunteer?}
\begin{itemize}
     \item $\sim$ 750k women applied for 200k jobs
     \item What is the effect of getting these jobs on the women?
     \item Identification from program selection rules, based on individual-level data on applicant characteristics
\end{itemize}
\end{itemize}
\end{frame}


\begin{frame}[plain]{Thank you}
\begin{itemize}
 \item Questions? Thoughts? Comments?
\end{itemize}
\end{frame}



\appendix
\mode<beamer>{  %  Options for use in slides only.
    \AtBeginSection[] {
    }
}


\renewcommand*{\refname}{} % This will define heading of bibliography to be empty, so you can...
\begin{frame}[plain,allowframebreaks]{Bibliography}
	\bibliographystyle{apacite}
\bibliography{tnll_bib}
\end{frame}

\begin{frame}[plain,label=representative]{Comparing TN ECE Baseline sample to NFHS - Household characteristics}
\centering
\resizebox{0.6\textheight}{!}{
\begin{tabular}{lccc}
\toprule
&   (1) 	&  (2)	&  (3) 	\\
& NFHS-V & Baseline & Difference    \\
& sample & sample &  \\
\midrule
\multicolumn{4}{l}{\textbf{Panel A: Assets}}\\
\input{Tables/DHS_Comparison_SES_Long}
\input{Tables/DHS_Comparison_SES_Long_N}
\bottomrule
\end{tabular}
}

\hyperlink{setting}{\beamergotobutton{Back}}
\end{frame}

\begin{frame}[plain,label=representative2]{Comparing TN ECE Baseline sample to NFHS - Household characteristics}
\centering
\resizebox{\textheight}{!}{
\begin{tabular}{lccc}
\toprule
&   (1) 	&  (2)	&  (3) 	\\
& NFHS-V & Baseline & Difference    \\
& sample & sample &  \\
\midrule
\multicolumn{4}{l}{\textbf{Panel B: Other characteristics}}\\
\input{Tables/DHS_Comparison_HH_charact}
\input{Tables/DHS_Comparison_HH_charact_N}
\midrule
\multicolumn{4}{l}{\textbf{Panel C: Parental education}}\\
\input{Tables/DHS_Comparison_short}
\input{Tables/DHS_Comparison_short_N}
\bottomrule
\end{tabular}
}

\hyperlink{setting}{\beamergotobutton{Back}}
\end{frame}

\begin{frame}[plain,label=attrition]{Comparing attriters to non-attriters}
\centering
\resizebox{0.8\textheight}{!}{
\begin{tabular}{lcccc}
\toprule
&   (1) 	&  (2)	&  (3) 	&  (4)	\\
& Surveyed a & Attrited & Difference   & Difference  \\
& at follow-up &  & (overall)  & (village FE) \\
\midrule
\input{Tables/Attrition}
\bottomrule
\end{tabular}
}

\hyperlink{setting}{\beamergotobutton{Back}}
\end{frame}





\begin{frame}[plain,label=inputs]{Difference in resources, inputs and child activities by maternal education}
\centering
\resizebox{0.85\textheight}{!}{
\begin{tabular}[t]{@{}l@{}l}
\toprule
\begin{tabular}[t]{lcccc}
&   (1) 	&  (2)	&  (3)  & (4) 		\\
& Primary & Incomplete   & Grade 12  &  (3)-(1)\\
&  or less & secondary  & or more &  \\
\midrule
\input{Tables/Inputs_mumed_cat}
\input{Tables/Inputs_mumed_cat_Nobs}
\end{tabular}
&
\begin{tabular}[t]{Hcc}
& (5) & (6) \\
 & Math & Tamil \\
 & value added & value added \\
\midrule
\input{Tables/ValueAdded_TercileTable}
\end{tabular}
\tabularnewline \bottomrule
\end{tabular}
}

\hyperlink{catchup}{\beamergotobutton{Back}}
\end{frame}


\begin{frame}[plain,label=inputs_itk]{Difference in resources, inputs and child activities, by \textit{(ITK)} attendance}
\centering
\resizebox{0.5\textheight}{!}{
\begin{tabular}{lcccc}
\toprule
&   (1) 	&  (2)	&  (3)  & (4)		\\
& Does not attend ITK & Attend ITK  & Difference & Difference   \\
&     & & (overall)&  (village FE)   \\
\midrule
\input{Tables/InputsITK}
\bottomrule
\end{tabular}
}

\hyperlink{itk_progressive}{\beamergotobutton{Back}}
\end{frame}


\begin{frame}[plain,label=itk_sensitivity_full]{Sensitivity of \textit{Illam Thedi Kalvi} estimates to including further inputs}
\centering
\resizebox{0.6\textheight}{!}{
    \begin{tabular}{lcccccccc}
    \toprule
&	\multicolumn{4}{c}{Math}&	\multicolumn{4}{c}{Tamil} \\
 \cmidrule(lr){2-5} \cmidrule(lr){6-9}
  & (1) & (2) & (3) & (4) & (5) & (6) & (7) & (8) \\
\midrule
\input{Tables/Table_ITKEffect_Senssitivity}
\bottomrule
\end{tabular}
}
\hyperlink{itk_effect}{\beamergotobutton{Back}}
\end{frame}

\begin{frame}[plain]{Raw distribution at baseline}
\begin{figure}[H]
\begin{center}
\includegraphics[width=0.8\textwidth]{Figures/Dist_Composite_age_2019.pdf} 
\end{center}
\footnotesize{\emph{Note: .}}
\end{figure}
\end{frame}

\begin{frame}[plain]{Raw distribution at the follow-up}
\begin{figure}[H]
\begin{center}
\includegraphics[width=0.8\textwidth]{Figures/Dist_Composite_age_2021.pdf} 
\end{center}
\end{figure}
\end{frame}



\end{document}